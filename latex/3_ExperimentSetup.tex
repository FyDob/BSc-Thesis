\section{Experiment Setup}\label{ch:experimentSetup}
\subsection{Evaluation}\label{evaluation}
Generating the words for the two experiments follows the procedure described by \cite{Bernardy2018} and will be explained in depth in the coming sections. Whereas the research put forth in his paper scrutinized the generative abilities of RNNs, I am investigating the models performing a classification task. As such, my training, test, validation and experiment data all consist of the same $1{:}1$ ratio of correct-to-incorrect words. Within the incorrect words, a distinction between superfluous opening or closing brackets is made, also at a ratio of $1{:}1$.

As such, a random guessing strategy would yield a baseline accuracy of $50\%$. A model is considered as having learned useful features from the training data if it scores above the baseline accuracy in the experiments. Furthermore, if the model has learned the underlying grammar of $D_{2}$, there should be no difference in accuracy for the two classes of incorrect words, as both of them do not belong to $D_{2}$, regardless of which bracket is replaced.

\subsection{Models}\label{models}
For the following experiments, the three RNN architectures described in Section \ref{neuralNetworkArchitectures} have been used. All models consist of an embedding layer, a single layer of size $n = \lbrace 2^{1}, 2^{2}, \dots, 2^{n} \rbrace$ and a dense layer of size $1$ with sigmoid activation. The activation value of the neuron in the dense layer acts as the output of the model: a value $\geq 0.5$ means the model classified the input as a correct word. The models were implemented in Tensorflow 2.0.\footnote{The source code can be found at \url {https://github.com/FyDob/BSc-Thesis}.}

All models were trained with the same parameters. The training data was received one word at a time, in batches of $512$. The loss was computed by binary cross-entropy, as is current standard for binary classification tasks. Furthermore, the Adam optimizer (\cite{Kingma2014}) was applied with a learning rate of $0.0001$. At the end of a training epoch, the models were evaluated for loss and accuracy on a validation set of $120{,}000$ words. The models were trained until their loss on the validation set did not lower by more than $0.0001$ for three consecutive epochs or for at most $100$ epochs. The models with the lowest validation loss were used for all experiments.

The models were trained on the same training data for both experiments. To answer the question of training data influence on model performance, three distinct sets of training data were used, yielding a total of $9 \times 3 \times 3 = 81$ (number of different hidden units $\times$ number of training corpora $\times$ number of different architectures) evaluated models.

\subsection{Corpus Construction}\label{corpusConstruction}
To investigate the influence of corpus composition on model performance, three corpora were created: a baseline corpus which is directly sampled from a subset of $D_{2}$, as well as two modifications of the baseline corpus: one impoverishing the training data from long-range dependencies (Low LRD) and one enriching the training data with more long-range dependencies (High LRD). The sampling and modification processes will be explained later in this section.

The experiments were explicitly designed to test the models' abilities to generalize based on the training data they encounter. As such, it is prudent to give consideration to which properties the training data might possess to facilitate or inhibit generalizability - properties such as length, maximum nesting depth (ND) and the maximum distance between a pair of opening and closing brackets (BD). ND is, in this case, defined as the highest number of unresolved open brackets preceding an open bracket in a given word (i.e. in the word \texttt{\{[\{\}]\}}, the square open bracket is at ND$=1$, and the curly open bracket is at ND$=2$, making the maximum ND of the word $2$). Maximum BD, then, is the highest number of characters between a pair of brackets in a word. In the previous example word, the maximum BD would be $4$. These measures are reported in Table \ref{tab:Corpora} in terms of averages and variance.

\begin{table}
	\begin{tabularx}{\textwidth}{@{}l*{10}{C}c@{}}
		\toprule 		
		\textit{Corpus} & \textit{Word Length} & \textit{maxND} & \textit{maxBD} \\ 
		\toprule 
		Baseline & $18.37$ ($6.36$) & $4.31$ ($1.22$) & $13.00$ ($16.02$) \\
		High LRD & $18.67$ ($4.75$) & $5.12$ ($1.04$) & $16.67$ ($4.75$) \\
		Low LRD & $17.54$ ($8.04$) & $3.92$ ($0.98$) & $10.58$ ($9.02$) \\
		\bottomrule
	\end{tabularx}
	\caption[Training corpora properties]{Properties of the three corpora the models were trained on, reported in averages (variance in brackets).}
	\label{tab:Corpora}
\end{table}

Furthermore, the training corpora were chosen to be a small slice of a comparatively large subset of $D_{2}$. To facilitate generalization, the training corpora consist of words of varying length. As discussed in Section \ref{relatedWorks}, previous works largely utilized similarly small language subsets and achieved encouraging results. For a discussion of Experiment $1$ and $2$ on a training corpus consisting of a majority of the target language, see \cite{Bernardy2018}.

In determining an eligible maximum length, a known fact about the size of $D_{n}$ subsets was utilized: a Dyck language $D_{n}$ contains $n^mC_{m}$ words of length $2m$, where $C_{m}$ is the $m$-th Catalan number (\cite{Skachkova2018}). It follows that a maximum length limit of $2m$ produces a set of size $\sum_{i=2}^{2m}n^{i}C_{i}$. For example, a maximum length of 20 in $D_{2}$ ($D_{2}^{\leq 20}$) yields $20{,}119{,}506$ words, which is a sufficiently large subset to sample from. The words were generated following the probabilistic grammar set forth by \cite{Sennhauser2018}.
\begin{align*}
	S &\rightarrow Z \; S \; \vert \; Z \\
	Z &\rightarrow B \; \vert \; T \\
	B &\rightarrow [ \; S \; ] \; \vert \; \lbrace \; S \; \rbrace \\
	T &\rightarrow [ \; ] \; \vert \; \lbrace \; \rbrace
\end{align*}
The production $Z \rightarrow B$ branches, whereas $S \rightarrow Z \; S$ concatenates two smaller Dyck words. This representation provides a good intuition for understanding the merit of Experiment 1. The probabilities with which the rules were applied are calculated as follows, with alternative rules of course being applied with the complementary probability:
\begin{align*}
	P_{\text{branch}} &= r_{\text{branch}} \cdot s(l) \quad \text{with } r_{\text{branch}} \sim U(0.7,1.0) \\
	P_{\text{concat}} &= r_{\text{concat}} \cdot s(l) \quad \text{with } r_{\text{concat}} \sim U(0.7,1.0) \\
	s(l) &= \min(1, -3 \cdot \frac{l}{n} + 3)
\end{align*}
with $l$ being the number of already generated non-terminal characters and $n$ the maximally desired length of the word. $r_{\text{branch}}$, $r_{\text{concat}}$ and $l$ were sampled at every step of word generation.

Following this process, $500{,}000$ words in $D_{2}^{\leq 20}$ were generated. These words served as the basis for creating the three corpora. To create the Low LRD corpus, all words with a maximum bracket distance higher than $10$ were modified\footnote{This cut-off point was chosen as it significantly reduces the average maximum bracket distance without creating too many duplicates.} by first identifying the bracket pair with the highest bracket distance, then simply moving the opening bracket from its original position to the position right before the closing bracket. (i.e. \texttt{\{[\{\}]\}} becomes \texttt{[\{\}]\{\}}). This has the largest impact on bracket distance throughout the corpus, while ensuring grammaticality of the resulting word. The resulting set of long-range impoverished words was merged with all unmodified words, deleting all duplicates.

The High LRD corpus was created in a similar way: First, all words with a bracket distance lower than $19$ were identified.\footnote{The same considerations as for the Low LRD corpus cut-off apply.} Then, the first pair of neighbouring closing brackets is found and deleted. The remaining word is wrapped in a randomly chosen pair of brackets, creating the longest possible bracket distance between the two (i.e. \texttt{\{[\{\}[]]\}} becomes \texttt{\{\{[[]]\}\}}). The resulting set was merged with the unmodified words the same way as the Low LRD set.


Finally, the corpora were filled with $500{,}000$ non-words obtained by corrupting the correct words in $D_{2}^{\leq 20}$. For one half of the words, a random opening bracket was replaced with a random closing bracket, while a random closing bracket was replaced with a random opening bracket for the other half.

In total, all corpora consist of $1{,}000{,}000$ samples, of which $50\%$ are incorrect.

\subsection{Experiment 1: Long-Range Dependency}\label{LRD}
For this experiment, the test set consisted of $1{,}000{,}000$ samples of length $1+18+18+1=38$, half of which were correct Dyck words. They were created by picking two random Dyck words $w_{1}, w_{2} \in D_{2}^{=18}$ from the base corpus, concatenating them and wrapping the result in a randomly selected pair of matching brackets as follows:
\[
	w_{\text{LRD}} = O_{n}w_{1}w_{2}C_{n}
\]
To generate incorrect samples, the generated correct LRD words were corrupted in the same way as for the training corpora, yielding $250{,}000$ incorrect LRD words with a superfluous opening or closing bracket each.

While $w_{1}$ and $w_{2}$ might have been seen in training (for models trained on the base corpus), the resulting word most certainly has not been observed. Neither could the model possibly have encountered a long-range dependency spanning $36$ characters between the opening and closing bracket. As such, a high classification accuracy serves as a strong indication of the model having learned to generalize to longer, non-concatenated Dyck words.\footnote{While the infixed sub-words are indeed concatenated, $w_{\text{LRD}}$ cannot be created by concatenating two shorter words due to being wrapped by a matching bracket pair.} I report model performance on Experiment 1 in terms of accuracy, precision, recall and F1 score.

\subsection{Experiment 2: New Depths}\label{ND}
To investigate how well a model performs on predicting brackets on a nesting level deeper than anything included in training, another test set was constructed. Since Experiment 1 already investigates Long-Range Dependency (LRD), this corpus was designed so its results are confounded as little as possible by LRD performance.

For this task, the test set consisted of  $1{,}000{,}000$ samples of length $30$, half of which were correct Dyck words. First, $500,000$  correct words were chosen at random from the base corpus. Then, they were wrapped by a prefix of five randomly chosen opening brackets and a suffix of the corresponding closing brackets as follows:
\[
	w_{\text{DN}} = O_{n}O_{n}O_{n}O_{n}O_{n}wC_{n}C_{n}C_{n}C_{n}C_{n}
\]
Generation of incorrect samples was done in accordance to Experiment 1 and corpus creation.

This process still has the model extrapolate beyond the length of the training words, while increasing all present nesting depths by $5$. This is analogous to center embedding in natural language - processing increasing nesting levels is more complicated than processing a flat structure. A high classification accuracy in Experiment 2 indicates a capability to generalize to repeated application of grammar rules beyond what was seen in the training set. As such, it implies an understanding of the $D_{2}$ grammar. I report model performance on Experiment 2 in terms of accuracy, precision, recall and F1 score.